\newenvironment{abstractpage}
  {\cleardoublepage\thispagestyle{empty}}
  {\vfill\cleardoublepage}
\newenvironment{abstract}[1]
  {\bigskip
   \begin{center}\bfseries#1\end{center}\small\leftskip=0.5cm\rightskip=0.5cm}
  {\par\bigskip}

\providecommand{\keywords}[2]{\footnotesize\textbf{\textit{#1:}} #2}

\begin{abstractpage}
\begin{abstract}{Abstract}

Most of the systems for network defense rely on evidence-based knowledge about the cyberattacks - threat intelligence. To prevent the attacks even before they happen, the firewalls and intrusion prevention systems rely on threat intelligence shared by the 

In this thesis, we introduce Fides. Fides is a generic trust model fine-tuned for sharing threat intelligence in highly adversarial global peer-to-peer networks of intrusion prevention agents.
We build Fides with the lessons learned from other state of the art trust models and optimized it for the broad spectrum of environments in the public global peer-to-peer network where anybody can join and leave at any time.


% Sharing the information about cybersecurity threats helps systems to prevent attacks even before they happen.
% However, anytime when a system acts on received threat intelligence, it trusts the source, that published said intelligence.
% Threat intelligence is nowadays mostly shared on public blocklists or centralized collaborative databases such as MISP.
% Centralized authorities then have full control over the data source and can decide what will be published and what will be censored.

%  
TBD \todo{add eng abstract}
\end{abstract}

\keywords{Keywords}{trust models, threat intelligence, collaborative network defense, intelligence sharing}

\vspace*{\fill}

\begin{abstract}{Abstrakt}
    TBD \todo{add cz absrakt}
    
\end{abstract}
\keywords{Klíčová slova}{TBD1, TBD2} \todo{add cz kws}

\end{abstractpage}
\thispagestyle{empty}

\cleardoublepage