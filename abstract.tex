\newenvironment{abstractpage}
  {\cleardoublepage\thispagestyle{empty}}
  {\vfill\cleardoublepage}
\newenvironment{abstract}[1]
  {\bigskip
   \begin{center}\bfseries#1\end{center}\small\leftskip=0.5cm\rightskip=0.5cm}
  {\par\bigskip}

\providecommand{\keywords}[2]{\footnotesize\textbf{\textit{#1:}} #2}

\begin{abstractpage}
\begin{abstract}{Abstract}

Most of the systems for network defense rely on evidence-based knowledge about the past cyberattacks - threat intelligence. To prevent the attacks even before they happen, the firewalls and intrusion prevention systems rely on the shared threat intelligence generated by other systems.
Such intelligence is usually shared via centralized public blocklists or databases where a single centralized authority has then full control over what is going to be published and what is going to be censored. In addition to that, it is a central point of failure.

To mitigate a central point of failure, one can use peer-to-peer networks to share the threat intelligence. However, because these networks are open to anyone including malicious actors, the peers need to be able to determine to who to trust and which data is better to discard.

In this thesis, we introduce Fides. Fides is a generic trust model fine-tuned for sharing threat intelligence in highly adversarial global peer-to-peer networks of intrusion prevention agents.
We build Fides with the lessons learned from other state of the art trust models and optimized it for the broad spectrum of environments in the peer-to-peer network where anybody can join and leave at any time.
Fides is able to identify peer's membership in organizations and trust their information accordingly when utilizing this knowledge, we prove that this knowledge improves 


\end{abstract}

\keywords{Keywords}{trust models, threat intelligence, collaborative network defense, intelligence sharing}

\vspace*{\fill}

\newpage
\begin{abstract}{Abstrakt}
    TBD \todo{add cz absrakt}
    
\end{abstract}
\keywords{Klíčová slova}{TBD1, TBD2} \todo{add cz kws}

\end{abstractpage}
\thispagestyle{empty}

\cleardoublepage