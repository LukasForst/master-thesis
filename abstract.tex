\newenvironment{abstractpage}
  {\cleardoublepage\thispagestyle{empty}}
  {\vfill\cleardoublepage}
\newenvironment{abstract}[1]
  {\bigskip
   \begin{center}\bfseries#1\end{center}\small\leftskip=0.5cm\rightskip=0.5cm}
  {\par\bigskip}

\providecommand{\keywords}[2]{\footnotesize\textbf{\textit{#1:}} #2}

\begin{abstractpage}
\begin{abstract}{Abstract}




% Sharing the information about cybersecurity threats helps systems to prevent attacks even before they happen.
% However, anytime when a system acts on received threat intelligence, it trusts the source, that published said intelligence.
% Threat intelligence is nowadays mostly shared on public blocklists or centralized collaborative databases such as MISP.
% Centralized authorities then have full control over the data source and can decide what will be published and what will be censored.

%  
TBD \todo{add eng abstract}
\end{abstract}

\keywords{Keywords}{trust models, threat intelligence, collaborative network defense, intelligence sharing}

\vspace*{\fill}

\begin{abstract}{Abstrakt}
    TBD \todo{add cz absrakt}
    
\end{abstract}
\keywords{Klíčová slova}{TBD1, TBD2} \todo{add cz kws}

\end{abstractpage}
\thispagestyle{empty}

\cleardoublepage