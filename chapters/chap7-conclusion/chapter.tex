\chapter{Conclusion}
\label{ch:conclusion}

In this thesis, we proposed, designed, and implemented \textbf{Fides}, a generic trust model for highly adversarial global peer-to-peer (P2P) networks of intrusion prevention system (IPS) agents, fine-tuned for sharing security threat intelligence. Fides implementation allows IPS agents to cooperate against malicious actors, also taking care of the security and privacy of the local environment. 
Although Fides was designed for threat intelligence sharing, it is generic and can be easily modified to share any other type of data. 

Fides is the first trust model designed for global P2P networks of defenders that considers memberships into organisations


Fides was implemented and evaluated by simulating thousands of scenarios to find the best trust model parameters for each of them.

We conclude that Fides is a viable solution for managing trust relationships in peer-to-peer networks and it is robust enough to work under highly-adversarial situations. Even in these situations, with 75\% of the peers malicious, it can correctly identify the truth of the shared threat intelligence data, and also identify the truth about how much to trust other peers.

Thanks to all the configurations and optimizations that Fides has for different organizations, we believe that Fides enables the Slips Intrusion Prevention System adoption across a broader spectrum of organizations.
We believe that Slips, combined with Fides, will enable more possibilities for sharing threat intelligence over the internet and improve the overall security of the networks that use Slips for the defense against intruders.


\include{chapters/chap7-conclusion/future_work}