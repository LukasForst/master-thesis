\chapter{Conclusion}
\label{ch:conclusion}

In this thesis, we proposed, designed, and implemented \textbf{Fides}. Fides is a generic trust model fine-tuned for sharing security threat intelligence in highly adversarial global peer-to-peer networks of intrusion prevention agents.
Fides enables threat intelligence sharing in global peer-to-peer networks of Intrusion Prevention Agents agents and allows them to cooperate against malicious actors without compromising the security of the local environment. 
Although Fides is fine-tuned for threat intelligence sharing, it is generic and can be easily modified to share any other IoC data.

We then evaluated the results of the experiments and the trust model in the Chapter~\ref{ch:results}, where we discovered that with a particular setup and under a condition that Fides talks to at least 25\% of pre-trusted peers, it could eventually identify targets correctly no matter how the rest of the peers behave.

Results have shown that Fides is a viable solution that tackles the problem of trust relationships in sensitive environments for threat intelligence sharing. 
We proved that Fides performs well and thus can be used to extend the existing implementation of Slips as a new module for sharing threat intelligence in the global peer-to-peer network.

Thanks to all the configurations and optimizations that Fides brings for usage inside the organizations, we believe that Fides enables the adoption of Slips Intrusion Prevention System across a broader spectrum of organizations.
We believe that Slips, combined with Fides, will enable more possibilities for sharing threat intelligence over the internet and improve the overall security of the networks that use Slips for the defense against intruders.


\include{chapters/chap7-conclusion/future_work}