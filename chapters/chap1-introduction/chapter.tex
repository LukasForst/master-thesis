\chapter{Introduction}
\label{ch:introduction}

% Structure of this chapter
% The why, the problem being solved
% What others did before
% The gap, what was missing
% Your proposal
% Your results
% Discussion on results/conclusions
% Your contributions


The name of the trust model - Fides. URL - \href{https://github.com/LukasForst/fides}{Github}.
Named after \href{https://en.wikipedia.org/wiki/Fides_(deity)}{Fides} - goddess of trust and good faith.

So far this document is just set of random paragraphs and notes for the future.

\section{Possible One Liners}
\begin{itemize}
\item An algorithm for the peer-to-peer network of IPS agents that evaluates agents’ behavior and data to assign them a trust value - essentially saying how much the local agent can trust another remote IPS.
\item A trust model for the peer-to-peer network of IPS agents that evaluates agents’ behavior.
\item A trust model for the global peer-to-peer network of IPS agents in highly adversarial environment.
\item Fides, Trust Model for Highly Adversarial Global Peer-To-Peer Networks of Intrusion Prevention Systems.
\item Fides, Trust Model for Global Peer-To-Peer Networks of Intrusion Prevention Systems.
\end{itemize}

\section{Problems we are solving}
\begin{enumerate}
\item How do we determine who and how much to trust in global peer-to-peer network, where peers can join and quickly leave and there is a possibility that all connected peers are adversarial.
\item Given that peers can join and leave any time, we will have problem with cold start. We need to come up with a way how can newcomers safely gain initial trust fast.
\item The main (but not only) purpose of the project is to be able to share threat intelligence over the network. How do we aggregate said threat intelligence into a single aggregated network opinion?
\item As a Slips administrator, I want to be able to share data only with (or trust incoming data only from) specific peers or specific organizations.
\end{enumerate}