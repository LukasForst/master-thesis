\section{Cold Start Problem}
\label{sec:cold-start-problem}
A dynamic and global environment such as a global peer-to-peer network is open to anyone since any peer can freely join and leave. Because of that, the local peer will encounter many other peers that were not seen before. Therefore, the trust model does not have any information about their reliability or how much it can trust them. 
New benign peers need to be \textit{somehow} trusted by the local peer in order to be a useful part of the network. However, the local peer also needs to be able to discover new malicious peers that are trying to gain its trust.

The problem of how to know something about a new entity in order to quickly work better is called the \textit{Cold Start Problem}~\cite{christensen2014hybrid}. For Fides it means how to compute a good initial trust for new unknown peers. 

There are multiple ways how to approach this issue, and we identified the following potential solutions. 
Fides has reference implementation for all of the following approaches and combines them according to provided configuration with the aim to achieve the best results for cold starts with malicious peers and behavior in mind.

\subsection{Static Initial Trust}
\label{subsec:static-initial-trust}
In this approach, whenever the trust model encounters a new peer, it assigns a static value as an initial trust. What the value depends on the specific implementation of the trust model.

For example, in \textbf{Dovecot} trust model, every peer starts with the trust of $1$ (highest possible), and various interactions can lower the trust in the peer to $0$. In other words, the trust model considers new peers honest from the beginning, and only during this time their reputation can be lowered when they perform incorrect interactions or are discovered as a malicious peer.

On the other hand, \textbf{Sality} botnet uses \textit{goodcount} as a counter of interactions with any other peer, the higher the \textit{goodcount}, the higher trust the peer has for the local peer. The goodcount for each new peer starts with $0$. Meaning, that the botnet does not trust fresh peers at all and they can gain trust only by following the protocol which depends on a number of good interactions and time.

Static initial trust is easy to implement, but it somewhat requires assumptions about the network. If the network is considered \textit{mostly being}, it might be safe to use an initial trust of $1$, however for highly adversarial networks using an initial trust of $1$ might be dangerous and it is better to use $0$. 
On the other hand, using low initial trust and no mechanism how to gain more trust fast means, that the being peers that joined recently, don't affect the final decisions of the model even though they might have useful information about adversaries.

Static initial trust is supported by Fides as a form of fallback when no other cold start technique is used. The administrator provides a configuration that contains the initial reputation for each new peer.

\subsection{Pre-Trusted Peers}
\label{subsec:pre-trusted-peers}
In a case, when the peer-to-peer network protocol allows peers to prove their identity, or to prove membership in some group, the trust model can utilize this knowledge and assign higher or lower trust.

The network layer, designed for Slips and Fides, supports this\cite{nl} and provides a cryptographically-secure way how to identify a single peer in the network and its membership in an organization.
This allows administrators to \textit{pre-trust} peers or the whole organization - by assigning them either initial reputation or service trust directly.

The Fides configuration allows administrators to specify static initial reputation for the specific peer or for all peers from the specific organization. 
This means that whenever a new peer joins the network and it is pre-trusted, it gains the initial reputation specified in the configuration.
During this time, it interacts with the local peer and provides threat intelligence, the data are evaluated and the trust model decides how much it trusts them (assigns service trust) based on the reputation and the quality of the data. In detail is this process described in \ref{sec:interaction-evaluation-strategies}.

Another option, the administrator can use, is to enforce the service trust for the peer for the time being. This effectively means that the trust model will not evaluate any data received from the pre-trusted peer and directly assigns them service trust from the configuration. This configuration for the Fides is called \textit{enforceTrust}.

Option to pre-trust peers solves the cold start problem for specific peers and organizations, as they will start with/keep a high reputation.
Which organization or which peer to trust or not is entirely on the administrator of the trust model. The inspiration for whom to trust provides, for example, Tor and their directory authorities~\cite{torauth}.
However as the administrator needs to know the identity of the peers or organization, it does not solve the cold start problem globally for all peers.

\subsection{Recommendations}
\label{subsec:recommendations}
As the local peer might have multiple remote peers, that it \textit{trusts enough}, it could be able to utilize this relationship and ask remote peers \textit{what they think about newcomer}. 
In other words, whenever a local peer encounters a peer that wasn't seen before, they can ask for a recommendation on this peer from the local peer's most trusted remote peers.

The recommendation system introduces new attack vectors, that can be exploited by adversaries either by getting trust for the malicious peer or by lowering trust in honest peers that might have some threat intelligence about the malicious actor. 
These attacks are called \textit{bad-mouthing} and \textit{unfair praises} and we need to consider them and implement countermeasures.

Because of the possible attacks, the local peer should not solely rely on the network recommendations when computing the final service trust for the fresh peer. In case, when the recommending peers are malicious, it might skew the decisions of the local peer for the time being.
In order to solve this, when computing the final service trust for the remote peer, the local peer should take into account its own interaction with the peer as well as the received recommendations.

Moreover, the local peer should request recommendations only if it has \textit{enough} trusted remote peers, otherwise, it can expose itself to \textit{bad-mouthing} and \textit{unfair praises} attacks more easily.

\vspace{7mm}

Fides employs recommendation systems based on SORT \cite{sort} and combines it with the pre-trusted peers (\ref{subsec:pre-trusted-peers}) as well as with the static initial trust (\ref{subsec:static-initial-trust}) as a fallback when no other option is available due to constraints such as having not enough trusted peers.
The algorithm used for the recommendation system is explained in detail in section \ref{sec:computational-model}.