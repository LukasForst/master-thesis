\section{Computational Model of Fides}
\label{sec:computational-model}
This section describes how Fides determines to whom and how much it can trust other remote peers.
Our trust model expresses trust in a specif peer with metrics called \textit{service trust}.
Service trust is a value that describes how much the local peer can trust a \textit{specific} remote peer. 

In the following pages, we describe the process top-down starting with the most important parts - service trust - and then breaking it down into bits.
Note that there are two main ideas behind most of the equations. 

The first one, is that we want to robustly capture the average behavior of the peers. In order to do that, we will be computing the average behavior of the peers and then approximating the deviations from said behavior.

The second part compares and weights first-hand experience with the remote experience. 
First-hand experience is what happened between local and remote peers during the time they interacted. This can be, for example, threat intelligence sharing, file-sharing, or the results of the recommendation protocol.
Remote experience is what happened between one remote peer and another remote peer. In other words, first-hand experience for peer $j$ are actions between $j$ and $z$. Whenever $j$ shares information about these actions with peer $i$, for $i$ it is a remote experience.

\vspace{0.5cm}

\noindent
Table~\ref{tab:notation-computational-model} describes the most important notation we use in the following sections.

\begin{table}[ht]
\centering
\begin{tabular}{ c | m{20em} }
 $i$ & local peer, instance of Fides \\
 \hline
 $j$ & remote peer somewhere on the internet \\
 \hline
 $st_{i, j}$ & service trust - how much $i$ trusts $j$ that it provides good service \\
 \hline
 $r_{i, j}$ & $i$'s reputation value about $j$ \\
 \hline
 $rt_{i, j}$ & $i$'s recommendation trust about $j$ \\
 \hline
 $sh_{i, j}$ & size of $i$'s service history with $j$ \\
 \hline
 $s^{k}_{i, j}$ & $i$'s satisfaction value with interaction with peer $j$ in window $k$\\
 \hline
 $w^{k}_{i, j}$ & weight of $i$'s interaction with $j$ in $k$ \\
 \hline
 $f^{k}_{i, j}$ & fading effect of $i$'s interaction with $j$ in $k$ \\
\end{tabular}
\caption{Fides Computational Model Notation}
\label{tab:notation-computational-model}
\end{table}

\subsection{Service Trust}
\label{subsec:service-trust}
As outlined previously, service trust $st_{i, j}$ is a value that describes how much peer $i$ trusts that remote peer $j$ will provide a \textit{good service}~\cite{sort}.
We compute the $st_{i, j}$ in the Equation~\ref{eq:service-trust} by weighing local experience with peer's $j$ service, with the reputation $j$ got from the network when it was first seen by $i$.
The used weight is the size of the service interaction history $sh_{i,j}$ to global maximal history size $sh_{max}$.

\begin{equation}
\label{eq:service-trust}
    st_{i,j}=\frac{sh_{i,j}}{sh_{max}} \cdot \left(cb_{i,j} - \frac{1}{2} ib_{i,j} \right) +\left(1-\frac{sh_{i,j}}{sh_{max}}\right) \cdot r_{i,j}
\end{equation}

Equation~\ref{eq:service-trust} implies that the more interaction there was between peers $i$ and $j$, the bigger impact on $st_{i,j}$ it has. 
In other words, the more $i$ and $j$ interact the less $i$ relies on the reputation that $i$ computed from the values provided by the network, at the time when $j$ was seen for the first time by the peer $i$.

\subsection{Local Experience for Service Trust}
The first part of the Equation~\ref{eq:service-trust} contains \textit{competence belief} - $cb_{i,j}$, and \textit{integrity belief} - $ib_{i,j}$.
Both values are based solely on the history of the interactions that the peer $i$ experienced with the peer $j$.

\subsubsection{Competence Belief}
\textit{Competence belief} represents how much did peer $j$ satisfied local peer $i$ with the past interactions. We measure it as an average of interactions from the past~\cite{sort}.

\begin{equation}
\begin{split}
    cb_{i,j} &= \frac{1}{\beta_{cb}} \sum_{k=1}^{sh_{i, j}} s_{i,j}^{k} \cdot w_{i,j}^{k} \cdot f_{i,j}^{k} \\
    \beta_{cb} &= \sum_{k=1}^{sh_{i, j}} s_{i,j}^{k} \cdot w_{i,j}^{k}
\end{split}
\end{equation}

It holds that $0 \leq cb_{i,j} \leq 1$ and where $s^{k}_{i,j}$ is the evaluation of the interaction in window $k$, $w^{k}_{i, j}$ is the weight of the interaction (how important it was) and $f^{k}_{i,j}$ is the fading effect of that interaction. We describe $s^{k}_{i,j}$, $w^{k}_{i,j}$ and $f^{k}_{i,j}$ in Section~\ref{subsec:interaction-satisfaction}. 
$\beta_{cb}$ is the normalization coefficient that ensures that $cb_{i, j}$ stays within the interval of $0 \leq cb_{i,j} \leq 1$.

\subsubsection{Integrity Belief}
\textit{Integrity belief} $ib_{i,j}$ is a level of confidence in the predictability of future interactions~\cite{sort}. It is measured as a deviation from the average behavior $cb_{i,j}$.
Therefore, $ib_{i,j}$ is calculated as an approximation to the standard deviation of interaction parameters~\cite{sort}.

\begin{equation}
\begin{split}
    ib_{i,j} &= \sqrt{\frac{1}{sh_{i,j}} \sum_{k=1}^{sh_{i,j}}\left(s_{i,j}^{k} \cdot w_{i,j}^{\mu} \cdot f_{i,j}^{\mu} - cb_{i,j}\right)^{2}} \\
    f_{i,j}^{\mu} &= \frac{1}{sh_{i, j}} \sum_{k=1}^{sh_{i,j}} f^{k}_{i,j} \\
    w_{i,j}^{\mu} &= \frac{1}{sh_{i, j}} \sum_{k=1}^{sh_{i,j}} w^{k}_{i,j}
\end{split}
\end{equation}

It holds that $0 \leq ib_{i,j} \leq 1$.
The more consistent behavior peer $j$ has, the lower the $ib_{i,j}$ is. Consistency is a highly desired property as the local peer then has more precise estimates about the future behavior of the remote peer.

\subsection{Interaction Satisfaction}
\label{subsec:interaction-satisfaction}
$s^{k}_{i, j}$ is $i$'s satisfaction value with interaction with peer $j$ in window $k$~\cite{sort}.
We outlined before, that each interaction between two peers is evaluated, $s^{k}_{i, j}$ is a result of this evaluation of a single interaction between peers $i$ and $j$.
Because our trust model is generic, the evaluation function can be implemented differently for different data.
How did we design it and implemented it for threat intelligence is described in the Section~\ref{sec:interaction-evaluation-strategies}.

However, even with the computed interaction satisfaction value, not all interactions are the same. 
Some interactions are more important than others. 
Moreover, because peers can change their behavior, most recent interactions should be more important than the interactions that happened a long time ago.
That is why we include the weight of the interaction and the fading effect.

\subsubsection{Weight of the Interaction}
Because each interaction is different and its importance is different, we have $w^{k}_{i,j}$ that measures the importance~\cite{sort}.
The weight belongs to interval $0 \leq w^{k}_{i,j} \leq 1$ and Fides implements it as a discrete function of interaction type. 
For example, the weight of interaction when a remote peer shares the threat intelligence is higher than when the remote peer requests threat intelligence.


\subsubsection{Fading Effect}
\label{subsubsec:fading-effect}
Fading effect $f^{k}_{i,j}$ determines \textit{"how much does the algorithm forget"} as the algorithm prefers most recent interactions over past interactions and thus $f^{k}_{i,j}$ reduces the weight of the past interactions~\cite{sort}. 
$f^{k}_{i,j}$ is a \textit{non-increasing function} of interaction and time or an index of said interaction in history.

The actual implementation of the fading effect depends on the data the trust model is processing.
For example, SORT implements it as a decreasing linear function $f^{k}_{i,j} = \frac{k}{sh_{i,j}}, 1 \leq k \leq sh_{i,j}$~\cite{sort}.
However, in our case and after multiple iterations, we decided not to forget the interactions that the model remembers and rather have all interactions with the same impact.

\begin{equation}
    f^{k}_{i,j} =1
\end{equation}

\noindent
The way Fides computes $f^{k}_{i,j}$ might be changed in the future and implemented as a function of time, we discuss this in more detail as a part of the future work in Section~\ref{sec:future-work}.

\subsection{Reputation and Recommendations}
In order to mitigate the cold start problem outlined in Section~\ref{sec:cold-start-problem} and in the cases when there are no or few interactions between $i$ and $j$, the algorithm relies on $r_{i,j}$ - \textit{reputation value}~\cite{sort}.
$r_{i,j}$ is the second part of the service trust Equation~\ref{eq:service-trust} that introduces \textit{remote experience} to the service trust.

The \textit{reputation} value is computed from the \textit{recommendations} received from the remote peers. This value represents what remote peers think about another remote peer. However, this value is calculated by the local peer with respect, to how much it trusts each peer, that provided the recommendation.
When the local peer $i$ encounters remote peer $j$ for the first time and it does not have any data about its trustworthiness, $i$ can request recommendations on peer $j$ from $i$'s most trusted peers.
We denote a set of remote peers, that provided the recommendations as $T_{i}$.

\subsubsection{Requesting a Recommendation}
\label{subsubsec:requesting-recommendation}
The recommendation system built into Fides cannot be used in every scenario.
Because of the sensitive nature of the environment, the trust model was designed for, there are cases when it is dangerous to ask for recommendations.
This is mainly the case when there are \textit{not enough} peers that are \textit{trusted enough}.

SORT requests recommendations every time it encounters a new peer. The set of recommending peers is created by taking all known peers and selecting those that have higher than average service trust.
However, those can also be peers with trust as low as $0.001$. In a sensitive environment, which the peer-to-peer network of IPS definitely is, we do not want to get recommendations from peers, that have low trust at all.
Moreover, given the nature of Slips, we decided to combine a recommendation system based on SORT with static initial trust~(\ref{subsec:static-initial-trust}) and with pre-trusted peers~(\ref{subsec:pre-trusted-peers}).
This approach provides a more robust basis for a trust-sensitive environment and it helps us to mitigate the cold start problem~(\ref{sec:cold-start-problem}).

If the peer is part of a pre-trusted organization or it is pre-trusted itself, it inherits the configured reputation - $r_{i, j}$ from the configuration.
In this case, Fides does not engage the recommendation protocol at all, because the peer already has reputation $r_{i,j}$ assigned from the configuration and it was \textit{recommended} by the administrator.
Moreover, the administrator can choose if this value is \textit{frozen}, or not. 
\textit{Frozen Service Trust} configuration means, that the peer $j$ has in eyes of $i$ \textit{static service trust} $st_{i, j}$ - it will never change and whatever data peer $j$ sends to $i$ will not influence the $st_{i,j}$.
On the other hand, when this configuration is not selected, the peer's service trust is going to change during the time when it communicates with the local instance according to the data and interactions it provides.

In the case where the peer is not pre-trusted, Fides evaluates if it has \textit{enough} well-trusted peers that can be trusted to provide the correct recommendation. This value as well as a number of maximal peers used for recommendation is configurable.
In addition, the administrator can enforce that for the recommendation protocol, only the pre-trusted peers or the peers from pre-trusted organizations are used.

\subsubsection{Recommendation Response}
A single recommendation response from peer $z \in T_{i}$ about giving the recommendation to peer $i$ about peer $j$ contains the following data.
\begin{itemize}
    \item $cb_{z,j}$, $ib_{z,j}$ - summary of $z$'s interactions with $j$, competence belief and integrity belief
    \item $sh_{z,j}$ - service history size, number of interactions between $z$ and $j$ - the more interactions they had, then the $z$'s recommendation has more credibility
    \item $r_{z, j}$ - summary of recommendations that $z$ received on $j$
    \item $\eta_{z,j}$ - number of peers that provided recommendations for $j$ when $j$ was new to $z$ and their recommendation was used to compute $r_{z,j}$
\end{itemize}

$cb_{z,j}$, $ib_{z,j}$ are included in the recommendation in order to provide a view on what does $z$ think about $j$.
$sh_{z,j}$ and $\eta_{z,j}$ are included to indicate how much experience with $j$ does $z$ actually have. To determine to which extent is the $z$ sure about correctness of $cb_{z,j}$, $ib_{z,j}$, $r_{z, j}$ in the recommendation.
And also to protect the $z$'s recommendation trust in $i$'s eyes, if $cb_{z,j}$, $ib_{z,j}$, $r_{z, j}$ values are wrong, because $i$ inspects $sh_{z,j}$ and $\eta_{z,j}$ and does not penalize $z$ that much, if the history size or the number of original recommender are low.

\subsubsection{Computing Reputation}
\label{subsubsec:computing-reputation}
When the local peer receives all recommendations, it computes the reputation value $r_{i,j}$ as a weighed expected local experience ($ecb_{i,j}$, $eib_{i,j}$ - estimates about competence and integrity) from the remote peers with their remote experience ($er_{i,j}$ - estimate about reputation of said peer).

\begin{equation}
\label{eq:reputation-value}
\begin{split}
    r_{i, j}=\frac{\lfloor\mu_{sh}\rfloor}{sh_{max}} \cdot \left(ecb_{i,j}-\frac{1}{2} eib_{i, j}\right) + \left(1-\frac{\lfloor\mu_{sh}\rfloor}{sh_{max}}\right) \cdot er_{i,j}
\end{split}
\end{equation}

\noindent
The weight, used in the Equation~\ref{eq:reputation-value}, is the average of history sizes in all recommendations to $sh_{max}$, maximum interactions history size. 
We calculate $\mu_{sh}$ as follows.
\begin{equation}
    \mu_{sh} = \frac{1}{|T_{i}|} \sum_{z \in T_{i}} sh_{z, j}
\end{equation}
\noindent
Again, we are weighing local experience to remote experience. However, in this case, it is local for the remote peers that provided the recommendations.

\subsection{Remote Local Experience}
\label{subsec:remote-local-experience}
Similarly, when we compute the service trust in Equation~\ref{eq:service-trust}, we need to get competence and integrity belief.
However, while creating reputation value in \ref{eq:reputation-value} where the values are coming from the remote peers, we are trying to estimate those values received from the network.
For that reason, we call them \textit{estimated competence belief} - $ecb_{i,j}$ and \textit{estimated integrity belief} - $eib_{i,j}$.

\subsubsection{Estimated Competence Belief}
$ecb_{i,j}$ is estimation about competence belief made by $i$ about $j$. 
This value is computed from the received recommendations in combination with $rt_{i,z}$ - a \textit{recommendation trust} that $i$ has about $z$.
Similarly, as for service trust, we have a normalization coefficient $\beta_{ecb}$ that moves the resulting data to the correct interval.
It holds that $0 \leq ecb_{i,j} \leq 1$.

\begin{equation}
\label{eq:estimated-competence-belief}
\begin{split}
    ecb_{i,j} &= \frac{1}{\beta_{ecb}} \sum_{z \in T_{i}} \left(rt_{i, z} \cdot sh_{z, j} \cdot cb_{z, j}\right) \\
    \beta_{ecb} &= \sum_{z \in T_{i}} \left(rt_{i, z} \cdot sh_{z, j}\right)
\end{split}
\end{equation}

\noindent
Recommendation trust $rt_{i, z}$ is described in detail in Section~\ref{subsec:recommendation-trust-metric}.

\subsubsection{Estimated Integrity Belief}
Following the $ecb_{i,j}$, $eib_{i,j}$ is estimation about the integrity belief made by $i$ about $j$.
Equation~\ref{eq:estimated-integrity-belief} is almost similar, but we use $ib_{z,j}$ instead of $eb_{z,j}$.
This means that normalization coefficient $\beta_{eib} = \beta_{ecb}$.

\begin{equation}
\label{eq:estimated-integrity-belief}
\begin{split}
    eib_{i,j} &= \frac{1}{\beta_{eib}} \sum_{z \in T_{i}} \left(rt_{i, z} \cdot sh_{z, j} \cdot ib_{z, j}\right) \\
    \beta_{eib} &= \sum_{z \in T_{i}} \left(rt_{i, z} \cdot sh_{z, j}\right)
\end{split}
\end{equation}

\subsection{Remote Remote Experience}
Going back to Equation~\ref{eq:reputation-value} from Section~\ref{subsubsec:computing-reputation}, we use \textit{estimated reputation value} - $er_{i,j}$.
This value represents information that was created by the peers that are remote even for remote peer $j$. 
In other words, this information came from the \textit{second ring of trust} - from acquaintances of an acquaintance.

\begin{equation}
\label{eq:estimated-reputation}
\begin{split}
    er_{i,j} &= \frac{1}{\beta_{er}} \sum_{z \in T_{i}} \left(rt_{i, z} \cdot \eta_{z, j} \cdot r_{z, j}\right) \\
    \beta_{er} &= \sum_{z \in T_{i}} \left(rt_{i, z} \cdot \eta{z, j}\right)
\end{split}
\end{equation}

\subsection{Recommendation Trust Metric}
\label{subsec:recommendation-trust-metric}
Recommendation trust - $rt_{i,z}$ - is another metric that a peer calculates and stores. It expresses how much does $i$ trust that $z$ provides \textit{good recommendations}.
Even though one could theoretically use service trust $st_{i, z}$ for this,
we have another trust metric because there are peers that can provide very good data (service), but they are surrounded by bad peers or the other way around.
This also gives us the ability to have specialized nodes in the network that serves as a peers registry for organizations - a single node that only provides recommendations on peers.

We calculate the recommendation trust in a similar way as the service trust and reputation, but we use recommendation competence belief $rcb_{i, z}$, recommendation integrity belief $rib_{i,z}$ and reputation $r_{i, z}$.
This time, we use the weight $rh_{i,z}$, which is the size of the history of the recommendations provided by $z$ to $i$, and $rh_{max}$, the maximal size of said history.

\begin{equation}
\label{eq:recommendation-trust}
\begin{split}
    rt_{i, z} = \frac{rh_{i,z}}{rh_{max}} \left(rcb_{i,z} - \frac{1}{2} rib_{i, z} \right) + \left(1 - \frac{rh_{i,z}}{rh_{max}} \right) r_{i,z}
\end{split}
\end{equation}

\subsubsection{Recommendation Competence and Integrity Belief}
\label{subsubsec:recommendation-competence-integrity-belief}
Similarly for interactions, we use three different parameters for calculating the $rcb_{i, z}$ and $rib_{i,z}$. 
We use satisfaction $rs^{x}_{i, z}$, weight $rw^{x}_{i, z}$ and the fading effect $rf^{x}_{i, z}$. 
The parameters have the same background as described in Section~\ref{subsec:interaction-satisfaction}, but in this case, they are connected to recommendations instead of service.
We calculate $rcb_{i, z}$ as follows:

\begin{equation}
\begin{split}
    rcb_{i, z} &= \frac{1}{\beta_{rcb}} \sum_{x = 1}^{rh_{i, z}}\left(rs_{i,z}^{x} \cdot rw_{i, z}^{x} \cdot rf_{i,z}^{x}\right) \\
    \beta_{rcb} &= \sum_{x = 1}^{rh_{i, z}}\left(rw_{i, z}^{x} \cdot rf_{i,z}^{x}\right)
\end{split}
\end{equation}

\noindent
And for recommendation integrity we compute $rib_{i, z}$ as:
\begin{equation}
    rib_{i, z} = \sqrt{\frac{1}{rh_{i, z}} \sum_{x=1}^{rh_{i,z}} \left(rs_{i,z}^{x} \cdot rw_{i, z}^{\mu} \cdot rf_{i,z}^{\mu} - rcb_{i,z}\right)^{2}}
\end{equation}

\noindent
One more time, the computational model is trying to approximate average behavior in recommendations - $rcb_{i,z}$ - and then the deviation from such behavior - $rib_{i,z}$.

Fading effect $rf^{x}_{i, z}$ has similar properties as the fading effect for service trust described in Section~\ref{subsubsec:fading-effect}. It is a non-increasing function of a number of recommendations or a time. 
For the recommendations, Fides implements it exactly the same as for the service interactions.

\begin{equation}
    rf^{x}_{i, z} = 1
\end{equation}

\subsubsection{Evaluating Received Recommendation}
As outlined in Section~\ref{subsubsec:recommendation-competence-integrity-belief}, in order to evaluate a particular recommendation from remote peer $z$, we have satisfaction, weight, and the fading effect. 
We calculate the recommendation satisfaction $rs^{x}_{i,z}$ by comparing values from $z$'s recommendation $r_{z,j}$, $cb_{z,j}$, $ib_{z,j}$, with values that are the results of the the recommendation algorithm.
In other words, we compare each recommendation, with the aggregated values - $er_{i,j}$, $ecb_{i,j}$ and $eib_{i,j}$. This gives us an estimate of how off was the peer $z$'s recommendation from the final result of the recommendation algorithm.

\begin{equation}
\begin{split}
    rs^{x}_{i,z} = \frac{1}{3} \left[ \left(1-\frac{\left|r_{z, j} - er_{i,j}\right|}{er_{i,j}}\right) + \right. \\
    \left(1 - \frac{\left|cb_{z, j} - ecb_{i, j}\right|}{ecb_{i, j}}\right) + \\
    \left. \left(1 - \frac{\left|ib_{z, j} - eib_{i, j}\right|}{eib_{i, j}}\right)\vphantom{\frac{1}{2}}\right]
\end{split}
\end{equation}

We calculate the weight of recommendation $rw^{x}_{i,z}$ as a weighed sum of the proportion of the size of the service history between $z$ and $j$ with maximal service history size. And a number of peers that provided the initial reputations $\eta_{z,j}$ divided by a maximal number of possible recommending peers.
\begin{equation}
    rw^{x}_{i,z} = \frac{\lfloor\mu_{sh}\rfloor}{sh_{max}} \cdot \frac{sh_{z, j}}{sh_{max}} + \left(1 - \frac{\lfloor\mu_{sh}\rfloor}{sh_{max}}\right) \cdot \frac{\eta_{z,j}}{\eta_{max}}
\end{equation}
