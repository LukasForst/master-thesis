\section{Experiments Evaluation}
\label{sec:experiments-evaluation}

Each experiment is evaluated with a single value $x$.
The evaluation function (\ref{eq:simulation_evaluation_function}) weights the performance of detecting the correct target score with the detection of peer behaviors.
\begin{equation}
\begin{split}
    x = w \cdot \frac{1}{|\tau|} \sum_{T \in \tau}\left|GS_{T} - S^{k_{max}}_{T} \right| + \left(1 - w\right) \cdot \frac{1}{|P|} \sum_{j \in P}\left|\bar{b_{j}} - st^{k_{max}}_{i, j} \right|
\end{split}
\label{eq:simulation_evaluation_function}
\end{equation}

In the first part, we measure average target detection, or how close was score produced by the trust model is to the ground truth.
We use the following notation - $\tau$ is the set of targets in the simulation, $GS_{T}$ is the ground truth score of the target, $S^{k_{max}}_{T}$ is then the score for the given target computed by Fides at the end of the simulation.

The second half is the evaluation that defines how close was trust model's service trust value for the remote peer to the peer's real behavior in the simulation.
$P$ is the set of remote peers in the simulation, $st^{k_{max}}_{i, j}$ is the service trust that the local trust model (\textit{i}) had for the remote peer (\textit{j}) at the end of the simulation.
$\bar{b_{j}}$ is then the ground truth behavior of the remote peer and we compute it in the equation \ref{eq:ground_truth_peer_behavior}.

\begin{equation}
    \begin{split}
    \bar{b_{j}} &= \frac{1 + shift \cdot \mu^{b}_{s}}{2}
    \end{split}
    \label{eq:ground_truth_peer_behavior}
\end{equation}

Recall the description of the peers' behaviors from the section \ref{sec:peers-behavioral-patterns}, where each peer's behavior $b$ had $\mu^{b}_{s}$ that was used during threat intelligence sampling.
Because the sampled score is $<-1; 1>$ and service trust $<0; 1>$, we can not use the $\mu^{b}_{s}$ directly, but we need to scale it to the correct interval.
Moreover, as malicious and incorrect peers  do have $\mu^{b}_{s}$ on the opposite scale that the ground truth is, we need to shift it before normalizing it.
For that reason, $shift = -1$ for malicious and incorrect peers and $shift = 1$ for confident correct, and uncertain behaviors and thus the equation \ref{eq:ground_truth_peer_behavior}.

\section{Simulations Execution}
\label{sec:simulations-execution}
The simulations and experiments were designed to evaluate the trust model in multiple ways and environments.
In order to run arbitrary scenarios, we developed a framework, that allows us to simulate virtually any environment with various combinations of Fides configuration.

Unfortunately, it is not possible to run and evaluate all possible scenarios, as there are 14 different sets of parameters that can have many different values.
This leads to the state explosion and therefore we were unable to cover all existing scenarios. 
However, alongside the Fides implementation, we published the simulation framework as well so anybody can simulate their preferred scenarios.

In the end, we reduced possible values for all parameters to ones, that we consider most important and run 165 890 different combinations of the environment as well as setup parameters.
During the simulations, we generated over 25 GB of raw data.
In the next chapter (\ref{ch:results}) we describe how we evaluated the experiments and we learned about the trust model behavior in various environments.