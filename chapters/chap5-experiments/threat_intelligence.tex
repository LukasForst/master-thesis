\section{Sampling Threat Intelligence}
\label{sec:sampling-threat-intelligence}
Threat intelligence, which is being shared on the peer-to-peer network and is aggregated by Fides, is generated inside Slips by various modules.
Each module provides a score on its own and Slips aggregates these evaluations into a single value. 
This means, that threat intelligence is computed as a sum of independent random variables and that tends to follow the normal distribution. 
For that reason, we sample threat intelligence values from the normal distribution.

As peers have different behavior, we will sample threat intelligence provided by them every time when they are asked for it.
We will characterize the peer's behavior by the threat intelligence it provides, with respect to the baseline, the ground truth of if the target is being or malicious.

As described in the previous chapters, threat intelligence consists of a \textit{score} and \textit{confidence} in that score.
We use the notation $\mu_{s}$ for the mean threat intelligence score and $\sigma_{s}$ for the standard deviation of the score. 
Same as for score, we use $\mu_{c}$ for mean confidence and $\sigma_{c}$ for the standard deviation of the confidence. 

Fides also in some cases employ recommendation protocol, so in the simulations, the peers might be asked to provide recommendations about other peers.
They will follow their behavioral strategy when providing the data. 
Recall recommendation description from section \ref{subsec:recommendations}. A single recommendation response contains $cb_{k,j}$, $ib_{k,j}$, $sh_{k,j}$, $r_{k,j}$ and $\eta_{k,j}$. 
We will be sampling those from the normal distribution as well with the corresponding pairs of $(\mu_{cb}, \sigma_{cb})$, $(\mu_{ib}, \sigma_{ib})$, $(\mu_{sh}, \sigma_{sh})$, $(\mu_{r}, \sigma_{r})$ and $(\mu_{\eta}, \sigma_{\eta})$.
Every peer will provide recommendations based on his behavioral strategy with respect to the ground truth.

