\subsection{Cold Start Problem}
\label{subsec:cold-start-problem}
Dynamic and global environment such as global peer-to-peer network, used by Fides, is open to anyone and any peer can freely join and leave. Because of that, the local peer will encounter many peers that were not seen before. As there was no previous encounter with the peer, the trust model does not have any information about their reliability nor how much can trust them. 
New peers need to be able to trust from the local peer in order to be useful part of the network. However, the local peer needs to be able to discover malicious actors that are trying to gain its trust to misuse it. 

We call this \textit{Cold Start Problem} - how does the new peer gain initial trust from the network. There are multiple ways how to approach this issue, we identified following potential solutions. 
Fides has reference implementation for all of the following approaches and combines them according to provided configuration with aim to achieve best results for cold starts with malicious peers and behavior in mind.

\subsubsection{Static Initial Trust}
In this approach, whenever trust model encounters new peer, it assigns static value as an initial trust. What the value is depends on the specific implementation of the trust model.

For example, in \textbf{Dovecot} trust model, every peer starts with the trust of $1$ (highest possible) and various interactions can lower the trust in the peer to $0$. In other words, the trust model considers new peers honest from the beginning and only during the time their reputation can be lowered when they perform incorrect interactions or are discovered as a malicious peers.

On the other hand, \textbf{Sality} botnet uses \textit{goodcount} as a counter of interactions with any other peer, higher the \textit{goodcount}, the higher trust the peer has for the local peer. The goodcount for each new peer starts with $0$. Meaning, that the botnet does not trust fresh peers at all and they can gain trust only by following the protocol which depends on number of good interaction and time.

Static initial trust is easy to implement, but it somewhat requires assumptions about the network. If the network is considered \textit{mostly being}, it might be safe to use initial trust of $1$, however for highly adversarial networks using initial trust of $1$ might be dangerous and it is better to use $0$. 
On the other hand, using low initial trust and no mechanism how to gain more trust fast means, that the being peers that joined recently, don't affect the final decisions of the model even though they might have useful information about adversaries.

Static initial trust is supported by Fides as form of fallback, when no other cold start technique is used. Administrator provides configuration which contains initial reputation for each new peer.

\subsubsection{Pre-Trusted Peers}
In a case, when the peer-to-peer network protocol allows peers to prove their identity, or to prove membership of some group, the trust model can utilize this knowledge and assign higher or lower trust.

The network layer, designed for Slips and Fides, supports this\cite{nl} and provides a cryptographically-secure way how to identify single peer in the network and its membership in an organization.
This allows administrators to \textit{pre-trust} peers or the whole organization - by assigning them either initial reputation or directly service trust.

The Fides configuration allows administrators to specify static initial reputation for the specific peer or for all peers from the specific organization. 
This means that whenever new peer joins the network and it is pre-trusted, it gains initial reputation specified in the configuration.
During the time, it interacts with the local peer and provides threat intelligence, the data are evaluated and trust model decides how much it trusts them (assigns service trust) based on the reputation and the quality of the data. In detail is this process described in \ref{subsec:interaction-evaluation}.

Another option, the administrator can use, is to enforce the service trust for the peer for the time being. This effectively means that trust model will not evaluate any data received from the pre-trusted peer and directly assigns them service trust from the configuration. This configuration for the Fides is called \textit{enforceTrust}.

Option to pre-trust peers solves the cold start problem for specific peers and organizations, as they will start with/keep the high reputation.
Which organization or which peer to trust or not is entirely on the administrator of the trust model. The inspiration for whom to trust provides, for example, Tor and their directory authorities\cite{torauth}. 
% TODO: probably do not have it here, but it is great example, this refers to "super trusted we-know-you-and-have-had-many-beers-with-you Tor volunteers"
However as the administrator needs to know the identity of the peers or organization, it does not solve the cold start problem globally for all peers.

\subsubsection{Recommendations}
As the local peer might have multiple remote peers, that it \textit{trusts enough}, it could be able to utilize this relationship and ask remote peers \textit{what do they think about newcomer}. 
In other words, whenever local peer encounters peer that wasn't seen before, it can ask for recommendation on this peer from the local peer's most trusted remote peers.

The recommendation system introduces new attack vectors, that can be exploited by adversaries either by getting trust for the malicious peer or by lowering trust in honest peer that might have some threat intelligence about the malicious actor. 
These attacks are called bad-mouthing and unfair praises and we need to consider them and implement countermeasures.






















