\section{Computational Model of Fides}
\label{sec:computational-model}
This section describes how does Fides come up with a single most important value $st_{i,j}$ service trust. Service trust describes how much can local peer $i$ trust remote peer $j$.
Used algorithm, that computes $st_{i,j}$, is based on SORT\cite{sort} with modifications to fit our use case - a global peer-to-peer network for sharing threat intelligence. The modifications are based on the interaction evaluation strategies proposed in \ref{subsec:interaction-evaluation} and algorithms to solve cold start problem described in \ref{subsec:cold-start-problem}.

\subsection{Intuition}
In the following pages, we describe the process top-down starting with the most important parts - service trust - and then breaking it down to bits.
There are two main ideas behind the most of the equations. 

The first one is that we want to robustly capture average behavior of the peers. In order to do that, we will be computing average behavior and standard deviations from said behavior and normalizing them.

Secondly, we will be comparing and weighting first hand experience with the remote experience. 
First hand experience is what happened between local and remote peer during time they interacted. This can be, for example, threat intelligence sharing, file sharing or the results of recommendation protocol.
Remote experience is what happened between one remote peer and another remote peer. In other words, first hand experience for peer $j$ are actions between $j$ and $z$. When $j$ shares information about these action with peer $i$, for $i$ it is a remote experience.

\begin{table}[ht]
\centering
\begin{tabular}{ c | m{20em} }
 $i$ & local peer, instance of Fides \\
 \hline
 $j$ & remote peer somewhere on the internet \\
 \hline
 $st_{i, j}$ & service trust - how much $i$ trusts $j$ that it provides good service \\
 \hline
 $r_{i, j}$ & $i$'s reputation value about $j$ \\
 \hline
 $rt_{i, j}$ & $i$'s recommendation trust about $j$ \\
 \hline
 $sh_{i, j}$ & size of $i$'s service history with $j$ \\
 \hline
 $s^{k}_{i, j}$ & $i$'s satisfaction value with interaction with peer $j$ in window $k$\\
 \hline
 $w^{k}_{i, j}$ & weight of $i$'s interaction with $j$ in $k$ \\
 \hline
 $f^{k}_{i, j}$ & fading effect of $i$'s interaction with $j$ in $k$ \\
\end{tabular}
\caption{Fides Computational Model Notation}
\label{tab:notation-computational-model}
\end{table}

\vspace{5mm}

\noindent 
The table \ref{tab:notation-computational-model} describes the most important notation we use in the following sections.

\subsection{Service Trust}
One of the major goals of the algorithm is to compute the service trust $st_{i,j}$.
We do that by weighting local experience with peer's $j$ service, with the reputation, $j$ got when it connected to the $i$. The weight here is size of the service interaction history $sh_{i,j}$ to maximal history size $sh_{max}$.

\begin{equation}\label{eq:service-trust}
st_{i,j}=\frac{sh_{i,j}}{sh_{max}} \cdot \left(cb_{i,j} - \frac{1}{2} \cdot ib_{i,j} \right) +\left(1-\frac{sh_{i,j}}{sh_{max}}\right) \cdot r_{i,j}
\end{equation}

The equation implies that the more interaction there was, between $i$ and $j$, the bigger impact on $st_{i,j}$ it has. 
In other words, the more $i$ and $j$ interact the less $i$ rely on the reputation that $i$ computed from the values provided by the network, at the time when $j$ was seen for the first time.

First part of the equation contains $cb_{i,j}$ - \textit{competence belief} - and $ib_{i,j}$ - \textit{integrity belief}. \textit{Competence belief} represents how well $j$ satisfied the $i$ with the past interactions. We measure it as an average of interactions from the past.
\textit{Integrity belief} is a level of confidence in predictability of future interactions. $ib_{i,j}$ is then measured as deviation from the average behavior $cb_{i,j}$. \cite{sort}

In order to mitigate cold start problem outlined in \ref{subsec:cold-start-problem} and in cases when there are no or little interactions between $i$ and $j$, the algorithm relies on $r_{i,j}$ - reputation value.