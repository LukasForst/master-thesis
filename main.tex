\documentclass{article}

\usepackage[utf8]{inputenc}
\usepackage{amsmath} % advanced math
\usepackage[hidelinks]{hyperref} % hide ugly boxes around links + add urls support
\usepackage{indentfirst} % intend first paragraph
\usepackage{algpseudocode} % add pseudocode
\usepackage{algorithm} % fancier pseudocode
\usepackage{array} % better alignments of tables
\usepackage{todonotes} % fancy todos

\usepackage{biblatex}
\addbibresource{biblio.bib} % bibliography


\title{Fides, Trust Model for Highly Adversarial Global Peer-To-Peer Networks of Intrusion Prevention Systems}
\author{Bc. Lukáš Forst}
\date{2022}

\begin{document}

\maketitle

The name of the trust model - Fides. URL - \href{https://github.com/LukasForst/fides}{Github}.
Named after \href{https://en.wikipedia.org/wiki/Fides_(deity)}{Fides} - goddess of trust and good faith.

So far this document is just set of random paragraphs and notes for the future.

\section{Possible One Liners}
\begin{itemize}
\item An algorithm for the peer-to-peer network of IPS agents that evaluates agents’ behavior and data to assign them a trust value - essentially saying how much the local agent can trust another remote IPS.
\item A trust model for the peer-to-peer network of IPS agents that evaluates agents’ behavior.
\item A trust model for the global peer-to-peer network of IPS agents in highly adversarial environment.
\item Fides, Trust Model for Highly Adversarial Global Peer-To-Peer Networks of Intrusion Prevention Systems.
\item Fides, Trust Model for Global Peer-To-Peer Networks of Intrusion Prevention Systems.
\end{itemize}

\section{Problems we are solving}
\begin{enumerate}
\item How do we determine who and how much to trust in global peer-to-peer network, where peers can join and quickly leave and there is a possibility that all connected peers are adversarial.
\item Given that peers can join and leave any time, we will have problem with cold start. We need to come up with a way how can newcomers safely gain initial trust fast.
\item The main (but not only) purpose of the project is to be able to share threat intelligence over the network. How do we aggregate said threat intelligence into a single aggregated network opinion?
\item As a Slips administrator, I want to be able to share data only with (or trust incoming data only from) specific peers or specific organizations.
\end{enumerate}



\subsection{Terminology}
\textbf{The problem:} here I want to describe what terminology I'm using

\begin{itemize}
\item \textbf{Fides} - name of the trust model, performs all trust related computations
\item \textbf{service trust} - how much a trust model trust a peer to provide us a good service - it is not necessarily how much we trust the data we received from the peer as to the final computation we may include the information about peer's IP address from the Slips (what does Slips think about the IP address)
\item \textbf{target} - an IP address or domain - point of interest that can be source of network traffic and Slips has a capability to assign threat intelligence data to this object
\item \textbf{remote peer} - peer that is running somewhere on the internet and it is connected to the global peer-to-peer network
\item \textbf{local peer} - local instance of Slips that is connected to the global peer-to-peer network and runs instance of Fides
\end{itemize}

\subsection{Interaction Evaluations}
\label{subsec:interaction-evaluation}
In order to determine what remote peers are providing valuable data and what peers are are not, the local peer needs to be able to evaluate each interaction it had with the remote peer.
In general, there are two options how to approach this - either by designing evaluation that is protocol aware (meaning, that it understands the protocol and the data that are two peers sharing with each other), or by having an evaluation function that does not need to understand the protocol and can be used for any data.

We choose to implement both approaches and they are described in the following sections. In order to evaluate which strategy is better in what scenarios, we designed and run many simulations - their results are described in section \ref{sec:simulations-and-evaluations}.

% TODO: maybe do not use S and C when referring to score and confidence as we use "s" for satisfaction
We will use notation from table (\ref{table:interaction-eval})  when referring to peers and their interactions.
\begin{table}[h!]
\centering
\begin{tabular}{ c | m{20em} }
 $i$ & local peer \\
 \hline
 $j$ & remote peer \\
 \hline
 $T$ & target of network intelligence, domain or IP address \\
 \hline
 $k$ & evaluation window \\
 \hline
 $s^{k}_{i, j}$ & $i$'s satisfaction value with interaction with peer $j$ in window $k$\\
 \hline
 $S^{k}_{j, T}$ & score computed by the peer $j$ about target $T$ in window $k$ \\
 \hline
 $C^{k}_{j, T}$ & confidence, how much is the score correct \\
 \hline
 $S^{k}_{T}$ & aggregated score from all threat intelligence reports in window $w$ for target $T$ \\
 \hline
 $C^{k}_{T}$ & aggregated confidence
\end{tabular}
\caption{Interactions Symbols}
\label{table:interaction-eval}
\end{table}


\subsubsection{Evaluate all interactions with the same value}
\label{subsubsec:same-eval-for-all-interactions}
The strategy, that does not need to understand underlying data, their semantics nor their structure.
It is a naive approach when the trust model uses the same satisfaction value for all data it received. It does not check, if the data make sense (for example when all other peers but one are reporting that the IP address is malicious) and assigns all peers same satisfaction value $s^{k}_{i, j}$. 
The idea behind this algorithm is that when the peers are interacting for a longer time or have more interactions, they're more trustworthy.

This approach is for example used by the botnet \textbf{Sality} or by the \textbf{Dovecot} trust model. Fides implements it as $EvenTIEvaluation$ strategy with configurable satisfaction value and administrator can use this strategy if they see it as the most optimal.

The disadvantage of this approach is, that we do not penalize remote peers when they provide wrong data, because the evaluation method does not care nor understand the underlying data.
Because of that and in a case when the adversary gains the service trust of the model by following the protocol for longer time, it may significantly influence the aggregated score as the adversary has higher trust then other remote peers. If this happens, there is no way to automatically downgrade adversary's service trust.

\subsubsection{Use aggregated network intelligence for evaluation}
\label{subsub:distance-based-eval}
Because Fides is designed for the sharing and aggregating threat intelligence and understands the protocol that is being used, we can utilize this and penalize peers that are providing local peer with incorrect data.
The interaction evaluation is performed at the end of the threat intelligence sharing process, at that point, Fides already aggregated data and decided what is the aggregated network score and confidence. 
Thus, we can utilize aggregated values and use it as a base line. Then we compare it against every each remote peer's threat intelligence we received. This evaluation strategy is implemented in the Fides a as a $DistanceBasedTIEvaluation$.

% TODO: here we need to set correct indexes, because k-th interaction between local and peer i is not necessarily the same number as for the peer i+1 -> that means that for S_a we will need another number
% TODO: we need to move this to another section and to describe what is score and what is confidence
Suppose, that remote peer $j$ provided data about target $T$ to local peer $i$ in window $k$. Provided data consist of score and confidence - ($S^{k}_{j, T}$, $C^{k}_{j, T}$). Where score,  $-1 \leq S^{k}_{j, T} \leq 1$, indicates if the target is malicious ($-1$) or begin ($1$). The confidence $0 \leq C^{k}_{j, T} \leq 1$ on the other hand indicates, how much is the peer sure about its assessment of $S^{k}_{j, T}$.

In order to evaluate interaction between the local peer $i$ and remote peer $j$ we need to compute satisfaction value $s^{k}_{i, j}$. 
It holds that  $0 \leq s^{k}_{i, j} \leq 1$ - where $1$ means peer $i$ was satisfied with the interaction.
% TODO: [?] this equation can be more robust, we can for example, compute standard deviation from the S^{k}_{T} and then compare each S to nearest second quantile -> that way the equation is more robust
\begin{equation}
s^{k}_{i, j} = (1 - \frac{|{S}^{k}_{T} - S^{k}_{j, T}|}{2} \cdot C^{k}_{j, T}) \cdot C^{k}_{T}
\end{equation}

Where $S^{k}_{T}$ is final score aggregated across the reports from the peers, $C^{k}_{T}$ is aggregated confidence.

The problem in this evaluation algorithm are the situations when the aggregated confidence $C^{k}_{T}$ is close to $0$. In this case the algorithm will penalize all peers for providing any threat intelligence as the final $s^{k}_{i, j}$ is close to $0$. Another issue with this approach is that when a single honest peer has a unique information about an IP address or domain, which is significantly different then what other peers have, it is automatically penalized for not sharing the same opinion as the other peers. However, if the peer is trusted enough, it has higher impact on the aggregated value and it is not penalized too much.

\subsubsection{Use network intelligence only if the confidence is high enough}
\label{subsubsec:network-intelligence-conf-high-enough}
In order to compensate for the low confidence $C^{k}_{T}$ and penalizing all peers in algorithm explained in \ref{subsub:distance-based-eval}, this evaluation strategy considers $C^{k}_{T}$ value and employs  $DistanceBasedTIEvaluation$ only when the  $C^{k}_{T}$ is \textit{"high enough"}. In this case \textit{"high enough"} means higher then configured value by the Slips administrator - ${CT}$.
In a case when  $C^{k}_{T} < {CT}$, the algorithm fall backs to using $EvenTIEvaluation$, because it is not possible to distinguish between \textit{"good"} and \textit{"bad"} network intelligence due to low confidence of the decision. 
This strategy is implemented in Fides under the name $ThresholdTIEvaluation$.

% TODO: I'm not sure if we really need this schema
\begin{algorithm}
\caption{$ThresholdTIEvaluation$}\label{alg:threshold-ti-evaluation}
\begin{algorithmic}[1]
\State ${CT} \gets configuration$ \Comment{configuration provided by the administrator}
\If{$C^{k}_{T} < {CT}$}
	\State $s^{k}_{i, j} \gets EvenTIEvaluation()$
\Else
    \State $s^{k}_{i, j} \gets DistanceBasedTIEvaluation()$
\EndIf
\end{algorithmic}
\end{algorithm}

\noindent What should be the correct value for $CT$ from configuration is subject to evaluation in the simulations in section \ref{sec:simulations-and-evaluations}.

\subsubsection{Use local threat intelligence to evaluate network intelligence}
\label{subsub:use-local-threat-to-evaluate}
This approach uses similar equation for computing satisfaction value outlined in \ref{subsub:distance-based-eval}. However, the input is different - instead of comparing remote peer's ($j$ ) threat intelligence ($S^{k}_{j, T}$, $C^{k}_{j, T}$) to aggregated intelligence ($S^{k}_{T}$, $C^{k}_{T}$), we compare it to the threat intelligence of the local ($i$) Slips instance - ($S^{k}_{i, T}$, $C^{k}_{i, T}$). Thus the evaluation is following:

\begin{equation}
s^{k}_{i, j} = (1 - \frac{|{S}^{k}_{i, T} - S^{k}_{j, T}|}{2} \cdot C^{k}_{j, T}) \cdot C^{k}_{i, T}
\end{equation}

This approach is useful when local peer has enough information about the target, but it wants to verify the behavior of the remote peers.
To determine whether they are sending data that are somewhat correct. This strategy is implemented in Fides with name $LocalCompareTIEvaluation$.

\subsubsection{Utilize all available data in evaluation}
The goal of this strategy is to evaluate the received data with as much confidence as possible. 
In order to do that, we combine all previous strategies into one, where we utilize all available information into a single $s^{k}_{i, j}$ value.

We introduce new variables here - $p_{0}, p_{1}, p_{2}$ - which are essentially weights of the particular strategies. These weights are based on the confidence, that the strategy has in its own decision.
Note, that there is a hierarchy and order matters. 
In our case we decided to prefer decisions coming from strategy \ref{subsub:distance-based-eval}, then we add data from the \ref{subsub:use-local-threat-to-evaluate} and if the final decision still does not have confidence of $1$, we add static value configured by the administrator (noted as $S_{C}$). 
The last part - $S_{C}$ - simulates static strategy described in \ref{subsubsec:same-eval-for-all-interactions}.
% TODO should we use min or the detailed version?

\begin{equation}
\label{equation:strategies-weights}
\begin{split}
    p_{0} &= {C}^{k}_{T} \\
    p_{1} &= \frac{1 - {C}^{k}_{T} + {C}^{k}_{i, T} - |1 - {C}^{k}_{T} - {C}^{k}_{i, T}|}{2} \\
    p_{1} &= min(1 - {C}^{k}_{T}, {C}^{k}_{i, T}) \\
    p_{2} &= 1 - p_{0} - p_{1}
\end{split}
\end{equation}
The weights $p_{0}, p_{1}, p_{2}$ in \ref{equation:strategies-weights}, are designed to gather as much confidence as possible. $p_{0}$ is confidence of the aggregated network data, essentially saying how much is the network sure about the given score. 
$p_{1}$ is the confidence coming from the local IPS and the $p_{2}$ is the remaining confidence to $1$.

When we have the weights, we can compute final $s^{k}_{i, j}$ where we use strategies - $p_{0} \cdot \ref{subsub:distance-based-eval}$, $p_{1} \cdot \ref{subsub:use-local-threat-to-evaluate}$ and $p_{2} \cdot \ref{subsubsec:same-eval-for-all-interactions}$. 
\begin{equation}
\begin{split}
    s^{k}_{i, j} &= \\
    &p_{0} \cdot [(1 - \frac{|{S}^{k}_{T} - S^{k}_{j, T}|}{2} \cdot C^{k}_{j, T}) \cdot C^{k}_{T}] + \\
    &p_{1} \cdot [(1 - \frac{|{S}^{k}_{i, T} - S^{k}_{j, T}|}{2} \cdot C^{k}_{j, T}) \cdot C^{k}_{i, T}] + \\
    &p_{2} \cdot S_{C}
\end{split}
\end{equation}

\noindent
This strategy is implemented in Fides under a name $MaxConfidenceEvaluation$.


\subsection{Network Intelligence Aggregation}
Fides is a trust model designed for global peer-to-peer network that is created by instances of Slips.
It is designed to support Slips in detecting malicious actors on the network and enables threat intelligence sharing between peers of Slips instances.
Because Slips was designed to be as modular as possible, Fides is effectively running as a module which provides aggregated threat intelligence to Slips. 
In other words, Fides provides a view on what does the network think about some threat intelligence target.
This means that Fides needs to be able to aggregate elements of threat intelligence from remote peers into a single value that is then presented to Slips - ($S^{k}_{T}$, $C^{k}_{T}$) - a network opinion on given target $T$ in window $k$.

We need to be able to say, that some reports are better then others based on the service trust the local peer has in remote peer - $st^{k}_{i, j}$.
Thus we need to weight every report based on this trust.
Let $R^{k}_{i, T}$ denote a set of remote peers that provided threat intelligence ($S^{k}_{j, T}$, $C^{k}_{j, T}$) to a local peer $i$ and $ws^{k}_{j}$ weight of their report (computed by $i$ as a normalized service trust that it has in the remote peer).
Then we can compute aggregated score $S^{k}_{T}$ like.
\begin{equation}
\begin{split}
    ws^{k}_{j} &= \frac{1}{\sum_{{j}\in R^{k}_{i, T}} st^{k}_{i, j}} \cdot st^{k}_{i, j} \\
    S^{k}_{T} &= \sum_{{j}\in R^{k}_{i, T}} ws^{k}_{j} \cdot S^{k}_{j, T}
\end{split}
\end{equation}

We also need to compute aggregated confidence $C^{k}_{T}$ that expresses how confident we are about the $S^{k}_{T}$ that was computed in the previous step.
Again, we use service trust to compute this for each remote peer and normalize it. 
\begin{equation}
\begin{split}
    C^{k}_{T} &= \frac{1}{|R^{k}_{i, T}|} \cdot \sum_{{j}\in R^{k}_{T}} st^{k}_{i, j} \cdot C^{k}_{j, T}
\end{split}
\end{equation}

\noindent 
Aggregated score and confidence ($S^{k}_{T}$, $C^{k}_{T}$) is then sent to Slips.
Depending on the selected interaction evaluation strategy (described in \ref{subsec:interaction-evaluation}), these values can be further used to evaluate each interaction with the remote peers.

\subsection{Recommendation System}
\textbf{Problem}: cold boot, with the dynamically changing network, how do new peers get initial trust, how can trusted peers recommend initial trust. Describe how recommendation system based on SORT works. Describe how do we approach pre-configured and pre-trusted peers and organizations.

\subsubsection{Cold Start Problem}
Dynamic and global environment such as global peer-to-peer network, used by Fides, is open to anyone and any peer can freely join and leave. Because of that, the local peer will encounter many peers that were not seen before. As there was no previous encounter with the peer, the trust model does not have any information about their reliability nor how much can trust them. 
New peers need to be able to trust from the local peer in order to be useful part of the network. However, the local peer needs to be able to discover malicious actors that are trying to gain its trust to misuse it. 

We call this \textit{Cold Start Problem} - how does the new peer gain initial trust from the network. There are multiple ways how to approach this issue, we identified following potential solutions.

\subsubsection{Static Initial Trust}
In this approach, whenever trust model encounters new peer, it assigns static value as an initial trust. What the value is depends on the specific implementation of the trust model.

For example, in \textbf{Dovecot} trust model, every peer starts with the trust of $1$ (highest possible) and various interactions can lower the trust in the peer to $0$. In other words, the trust model considers new peers honest from the beginning and only during the time their reputation can be lowered when they perform incorrect interactions or are discovered as a malicious peers.

On the other hand, \textbf{Sality} botnet uses \textit{goodcount} as a counter of interactions with any other peer, higher the \textit{goodcount}, the higher trust the peer has for the local peer. The goodcount for each new peer starts with $0$. Meaning, that the botnet does not trust fresh peers at all and they can gain trust only by following the protocol which depends on number of good interaction and time.

Static initial trust is easy to implement, but it somewhat requires assumptions about the network. If the network is considered \textit{mostly being}, it might be safe to use initial trust of $1$, however for highly adversarial networks using initial trust of $1$ might be dangerous and it is better to use $0$. 
On the other hand, using low initial trust and no mechanism how to gain more trust fast means, that the being peers that joined recently, don't affect the final decisions of the model even though they might have useful information about adversaries.

Static initial trust is supported by Fides as form of fallback, when no other cold start technique is used. Administrator provides configuration which contains initial reputation for each new peer.

\subsubsection{Pre-Trusted Peers}
In a case, when the peer-to-peer network protocol allows peers to prove their identity, or to prove membership of some group, the trust model can utilize this knowledge and assign higher or lower trust.

The network layer, designed for Slips and Fides, supports this\cite{nl} and provides a cryptographically-secure way how to identify single peer in the network and its membership in an organization.
This allows administrators to \textit{pre-trust} peers or the whole organization - by assigning them either initial reputation or directly service trust.

The Fides configuration allows administrators to specify static initial reputation for the specific peer or for all peers from the specific organization. 
This means that whenever new peer joins the network and it is pre-trusted, it gains initial reputation specified in the configuration.
During the time, it interacts with the local peer and provides threat intelligence, the data are evaluated and trust model decides how much it trusts them (assigns service trust) based on the reputation and the quality of the data. In detail is this process described in \ref{subsec:interaction-evaluation}.

Another option, the administrator can use, is to enforce the service trust for the peer for the time being. This effectively means that trust model will not evaluate any data received from the pre-trusted peer and directly assigns them service trust from the configuration. This configuration for the Fides is called \textit{enforceTrust}.

Option to pre-trust peers solves the cold start problem for specific peers and organizations, as they will start with/keep the high reputation.
Which organization or which peer to trust or not is entirely on the administrator of the trust model. The inspiration for whom to trust provides, for example, Tor and their directory authorities\cite{torauth}. 
% TODO: probably do not have it here, but it is great example, this refers to "super trusted we-know-you-and-have-had-many-beers-with-you Tor volunteers"
However as the administrator needs to know the identity of the peers or organization, it does not solve the cold start problem globally for all peers.

\subsubsection{Recommendations}
As the local peer might have multiple remote peers, that it \textit{trusts enough}, it could be able to utilize this relationship and ask remote peers \textit{what do they think about newcomer}. 
In other words, whenever local peer encounters peer that wasn't seen before, it can ask for recommendation on this peer from the local peer's most trusted remote peers.

However, the recommendation system introduces new attack vectors, that can be exploited by adversaries either by getting trust for the malicious peer or by lowering trust in honest peer that might have some threat intelligence about the malicious actor. 
These attacks are called bad-mouthing and unfair praises. 














\section{Computational Model of Fides}
\label{sec:computational-model}
\textbf{Problem} - here we want to describe what we actually used and how to we determine real service trust and reputations across  the whole system, we also want to describe whole data flow, what we compute where and how do we get to final result

\vspace{10mm}

This section describes how does Fides come up with a single most important value $0 \leq st_{i,j} \leq 1$ service trust. Service trust describes how much can local peer $i$ trust remote peer $j$.
Used algorithm, that computes $st_{i,j}$, is based on SORT\cite{sort} with modifications to fit our use case - a global peer-to-peer network for sharing threat intelligence.

\begin{equation}
st_{i,j}=\frac{sh_{i,j}}{sh_{max}} \cdot \left(cb_{i,j} - \frac{1}{2} \cdot ib_{i,j} \right) +\left(1-\frac{sh_{i,j}}{sh_{max}}\right) \cdot r_{i,j}
\end{equation}

\section{Simulations and Evaluation}
\label{sec:simulations-and-evaluations}
this is section where we describe how we run the simulations and what are the results of the evaluations

These are some equations we will need in order to describe the SORT algorithm.


$$
f_{i j}^{\mu}=\frac{1}{s h_{i j}} \sum_{k=1}^{s h_{i j}} f_{i j}^{k}=\frac{s h_{i j}+1}{2 s h_{i j}} \approx \frac{1}{2} .
$$

$$
e r_{i j}=\frac{1}{\beta_{e r}} \sum_{p_{k} \in T_{i}}\left(r t_{i k} \cdot \eta_{k j} \cdot r_{k j}\right) .
$$

$$
r_{i j}=\frac{\left\lfloor\mu_{s h}\right\rfloor}{s h_{\max }}\left(e c b_{i j}-e i b_{i j} / 2\right)+\left(1-\frac{\left\lfloor\mu_{s h}\right\rfloor}{s h_{\max }}\right) e r_{i j}
$$

$$
r s_{i k}^{z}=\left(\left(1-\frac{\left|r_{k j}-e r_{i j}\right|}{e r_{i j}}\right)+\left(1-\frac{\left|c b_{k j}-e c b_{i j}\right|}{e c b_{i j}}\right)\right. \\
\left.+\left(1-\frac{\left|i b_{k j}-e i b_{i j}\right|}{e i b_{i j}}\right)\right) / 3
$$

$$
r w_{i k}^{z}=\frac{\left\lfloor\mu_{s h}\right\rfloor}{s h_{\max }} \frac{s h_{k j}}{s h_{\max }}+\left(1-\frac{\left\lfloor\mu_{s h}\right\rfloor}{s h_{\max }}\right) \frac{\eta_{k j}}{\eta_{\max }} .
$$

\begin{gather}
    r c b_{i k}=\frac{1}{\beta_{r c b}} \sum_{z=1}^{r h_{i k}}\left(r s_{i k}^{z} \cdot r w_{i k}^{z} \cdot r f_{i k}^{z}\right), \\
    r i b_{i k}=\sqrt{\frac{1}{r h_{i k}} \sum_{z=1}^{r h_{i k}}\left(r s_{i k}^{z} \cdot r w_{i k}^{\mu} \cdot r f_{i k}^{\mu}-r c b_{i k}\right)^{2},}
\end{gather}


$$
r t_{i k}=\frac{r h_{i k}}{r h_{\max }}\left(r c b_{i k}-r i b_{i k} / 2\right)+\frac{r h_{\max }-r h_{i k}}{r h_{\max }} r_{i k}
$$


\medskip

\printbibliography

\end{document}
